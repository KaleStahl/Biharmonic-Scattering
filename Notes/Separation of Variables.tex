\documentclass[]{article}
\usepackage[utf8]{inputenc}
\usepackage[left=0.5in, right = 0.5in, top = .75in, bottom = .75in]{geometry}
\usepackage[usenames,dvipsnames]{xcolor}
\usepackage{amsmath, amsfonts}
\usepackage{tikz}
\usepackage{comment}
\usepackage[english]{babel}
\usepackage{amsthm}
\usepackage{enumitem}
\usepackage{graphicx}
\usepackage{mathtools}
\usepackage{xstring}
\usepackage{fancyhdr}
\usepackage{subcaption}
\usepackage{float}
\usepackage[most,many,breakable]{tcolorbox}
\usepackage{hyperref}

\newtheorem{theorem}{Theorem}
\numberwithin{theorem}{section}
\newtheorem{corollary}{Corollary}[theorem]
\newtheorem{definition}{Definition}
\numberwithin{definition}{section}
\newtheorem{proposition}{Proposition}
\numberwithin{proposition}{section}
\newtheorem{lemma}{Lemma}
\numberwithin{lemma}{section}
\renewcommand{\theenumi}{\Alph{enumi}}
\numberwithin{equation}{section}

\newcommand{\bd}{\textbf}
\newcommand{\del}{\nabla}
\newcommand{\mbf}{\mathbf}
\newcommand{\inv}{^{-1}}
\newcommand{\sbf}{\boldsymbol}
\newcommand{\by}{\times}
\newcommand{\R}{\mathbb R}
\newcommand{\Rn}{{\mathbb R^n}}
\newcommand{\C}{\mathbb C}
\newcommand{\N}{\mathbb N}
\newcommand{\Q}{\mathbb Q}
\newcommand{\Z}{\mathbb Z}
\newcommand{\cM}{\mathcal M}
\newcommand{\normsubgp}{\unlhd}
\newcommand{\cN}{\mathcal N}
\newcommand{\cH}{\mathcal H}
\newcommand{\cS}{\mathcal S}
\newcommand{\cX}{\mathcal X}
\newcommand{\cY}{\mathcal Y}
\newcommand{\pcurl}{\nabla \times}
\newcommand{\pdiv}{\nabla \cdot}
\newcommand{\curl}{\mathrm{curl}\,}
\newcommand{\im}{\mathrm{im}\,}
\newcommand{\coim}{\mathrm{coim}\,}
\newcommand{\rank}{\mathrm{rank}\,}
\renewcommand{\ker}{\mathrm{ker}\,}
\newcommand{\trace}{\mathrm{tr}\,}
\renewcommand{\div}{\mathrm{div}\,}
\renewcommand{\Im}{\mathrm{Im}\,}
\renewcommand{\Re}{\mathrm{Re}\,}
\newcommand{\res}{\mathrm{Res}\,}
\newcommand{\End}{\mathrm{End}\,}
\newcommand{\Span}{\mathrm{Span}\,}
\newcommand{\stab}{\mathrm{stab}\,}
\newcommand{\orbit}{\mathrm{orbit}\,}
\renewcommand{\phi}{\varphi}
\newcommand{\supp}{\mathrm{supp}\,}
\renewcommand{\epsilon}{\varepsilon}
\renewcommand{\bar}{\overline}
\renewcommand{\hat}{\widehat}

\newcommand{\innerprod}[2]{\left\langle #1,\, #2\right\rangle}
\newcommand{\avg}[1]{\left\langle #1\right\rangle}
\newcommand{\pder}[3][]{%
	\IfEqCase{#2}{%
		{1}{\frac{\partial#1}{\partial #3}}%
	}[\frac{\partial^{#2}}{\partial #3^{#2}}]%
}
\NewDocumentCommand{\paren}{m}{%
	\left(%
	{#1}%
	\right)%
}
\NewDocumentCommand{\bracket}{m}{%
	\left[%
	{#1}%
	\right]%
}


\title{Separation of Variables}
\date{Spring 2025}
\author{Kale Stahl}

\pagestyle{fancy}
\fancyhf{}
\makeatletter
\lhead{\@title}
\chead{\@date}
\rhead{\@author}
\makeatother

\begin{document}
	
	%Makes fancy title
	\makeatletter
	\begin{center}
		\vspace{.25cm}
		{\centering \Large \bd \@title}\\
		\vspace{.5cm}
		{\large \@author}
		\vspace{.75cm}
	\end{center}
	\makeatother
	For the domain $\R^2\setminus B(1, 0)$, we wish to solve
	\begin{align}
		u^s = u_{pr} + u_{ev}\\
		\Delta u_{pr}+ k^2u_{pr} = 0\\
		\Delta u_{ev} - k^2 u_{ev} = 0
	\end{align}
	Where
	\begin{align}
		u_{pr}(r, \theta) &= \sum_p i^pa_pH^{(1)}_p(kr)e^{ip(\theta - \phi)}\\
		u_{ev}(r, \theta) &= \sum_p i^pb_pH^{(1)}_p(ikr)e^{ip(\theta - \phi)}
	\end{align}
	Then inside of $B(1, 0)$ we have 
	\begin{align}
		u = u_H + u_M\\
		\Delta u_{H}+ k^2\sqrt n u_{H} = 0\\
		\Delta u_{M} - k^2\sqrt n u_{M} = 0
	\end{align}
	where
	\begin{align}
		u_{H}(r, \theta) &= \sum_p i^pc_pJ_p(k\sqrt[4]{n}r)e^{ip(\theta - \phi)}\\
		u_{M}(r, \theta) &= \sum_p i^pd_pJ_p(ik\sqrt[4]{n}r)e^{ip(\theta - \phi)}
	\end{align}
	We get the system
	\begin{align}
		u^s - u = -u^i\\
		\partial_ru^s- \partial_r u^i = - \partial_r u^i\\
		\Delta u^s - \Delta u = - \Delta u^i\\
		\partial_r\Delta u^s - \partial_r \Delta u = \partial_r\Delta u^i
	\end{align}
	We wish to solve for the constants $a_p$, as 
	\begin{align}
		u^\infty = u^\infty_{pr}(\theta, \phi) \approx \sum_{|p| = 5}\frac{4}{i}a_pe^{ip(\theta -\phi)}
	\end{align}
	Simplifying each equation individually, we see
	\begin{align}
		u_{pr}+u_{ev} -(u_H + u_M) &= -u^i\\
		\sum_p \paren{i^pa_pH^{(1)}_p(kr)e^{ip(\theta-\phi)}+i^pb_pH^{(1)}_p(ikr)e^{ip(\theta - \phi)}-\paren{i^pc_pJ_p(k\sqrt[4]{n}r)e^{ip(\theta-\phi)}+i^pd_pJ_p(ik\sqrt[4]{n}r)e^{ip(\theta - \phi)}}} &= -\sum_pi^pJ_p(kr)e^{-ip(\theta - \phi)}\\
		a_pH^{(1)}_p(kr)+b_pH^{(1)}_p(ikr)-c_pJ_p(k\sqrt[4]{n}r)-d_pJ_p(ik\sqrt[4]{n}r) &= -J_p(kr)
	\end{align}
	\begin{align}
	\partial_r u_{pr}+\partial_r u_{ev} -(\partial_r u_H + \partial_r u_M) &= -\partial_r u^i\\
		\sum_p \paren{i^pka_pH^{(1)}_p(kr)e^{ip(\theta-\phi)}+i^pikb_pH^{(1)}_p(ikr)e^{ip(\theta - \phi)}-\paren{i^pc_pJ_p(k\sqrt[4]{n}r)e^{ip(\theta-\phi)}+i^pd_pJ_p(ik\sqrt[4]{n}r)e^{ip(\theta - \phi)}}} &= -\sum_pi^pJ_p(kr)e^{-ip(\theta - \phi)}\\
		a_pH^{(1)}_p(kr)+b_pH^{(1)}_p(ikr)-c_pJ_p(k\sqrt[4]{n}r)-d_pJ_p(ik\sqrt[4]{n}r) &= -J_p(kr)
	\end{align}
\end{document}